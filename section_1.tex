%%%%%%%%%% Computing Energy of Heat equations %%%%%%%%%%

\section{Heat Equation and its Energy}
 The Heath Equation is as following:
 \begin{equation}
     U_t = a{U}_{xx}
 \end{equation}
 and conditions are 
 \begin{align*}
     U(\pi,0) &= 0\\
     U(x,0) &= {U}_{0}(x)
 \end{align*}
     
 By Separation of Variable we have the solution of heat equation as
 \begin{equation}
     U(x,t) = \sum_{k=1}^{\infty} A_{k}(\sin{kx})e^{(-k^2t)}
 \end{equation}
 where
 \begin{equation}
     A_k = \frac{2}{\pi} \int_{0}^{\pi} U_{0}(x)(\sin{kx})dx
 \end{equation}
  
  Now we prove Energy for the system is decreasing i.e. System is dissipative.
 \begin{equation}
     U_t = a{U}_{xx}
 \end{equation}
Now we multiply the  $U(x,t)$ on both sides of Heat Equation.
\begin{equation}
     U*U_t = U*(a{U}_{xx})
\end{equation}
and Now proceed to integrate it over the interval $[0,\pi]$.
\begin{equation}
     \int_{0}^{\pi}(U*U_t - U*(a{U}_{xx}))dx = 0
\end{equation}
and now we change the term in which time derivative of U occurs as follows

\begin{equation}
     \int_{0}^{\pi}(U*U_t)dx = \frac{1}{2} \int_{0}^{\pi}\frac{\partial}{\partial{t}}U^{2}dx
\end{equation}

and we also know that integration is linear operator so we have
\begin{equation}
     \frac{1}{2} \int_{0}^{\pi}\frac{\partial}{\partial{t}}U^{2}dx - \int_{0}^{\pi}U*(a{U}_{xx})dx = 0
\end{equation}

and now by Liebniz rule we know that 

\begin{equation}
     \frac{1}{2} \int_{0}^{\pi}\frac{\partial}{\partial{t}}U^{2}dx = \frac{1}{2}\frac{d}{dt} \int_{0}^{\pi}U^{2}dx 
\end{equation}

and hence we get 
\begin{equation}
     \frac{1}{2}\frac{d}{dt} \int_{0}^{\pi}U^{2}dx  - \int_{0}^{\pi}U*(a{U}_{xx})dx = 0
\end{equation}

Now we integrate the second term i.e. 2nd order partial derivative with respect to $x$, by "Integration by Parts" as follows

\begin{equation}
     \int_{0}^{\pi}{U{U}_{xx}}dx = U*U_{x} \Big|_{0}^{\pi} - \int_{0}^{\pi}{U_x}{U_x}dx
\end{equation}


\begin{equation}
     \int_{0}^{\pi}{U{U}_{xx}}dx = U(\pi,t)*U_{x}(\pi,t) - U(0,t)*U_{x}(0,t) - \int_{0}^{\pi}{U_x}^{2}dx
\end{equation}

since by given conditions we know that 

\begin{align*}
    U(\pi,t) &= 0 \\
    U(0,t) &= 0 \\
\end{align*}

and final expression becomes
\begin{equation}
     \int_{0}^{\pi}{U{U}_{xx}}dx = -\int_{0}^{\pi}{U_x}^{2}dx
\end{equation}

and our main equation is:
\begin{equation}
     \frac{1}{2}\frac{d}{dt} \int_{0}^{\pi}U^{2}dx  =-a\int_{0}^{\pi}{U_x}^{2}dx
\end{equation}

Provided that $a>0$
\begin{equation}
     a\int_{0}^{\pi}{U_x}^{2}dx \geq 0
\end{equation}

Now as we know that Energy function  $E(t)$ is 

\begin{equation}
     E(t) = \frac{1}{2} \int_{0}^{\pi} U^{2}dx
\end{equation}

and so 

\begin{equation}
    \frac{d}{dt}E(t) \leq 0
\end{equation}

Hence it is proved that the time rate of Energy function is non-increasing so our given system is dissipative system.

Yaaro Yaari .... $=$ Hence Proved




%%%%%%%%%% Computing Energy of wave equations %%%%%%%%%%

\section{Wave Equation and its Energy}
The Wave Equation is following:

\begin{equation}
     U_{tt} - a{U}_{xx} = 0
\end{equation}

and the conditions are 
\begin{align*}
    U(0,t) &= 0\\
    U(\pi,t) &= 0\\
    U(x,0) &= U_{0}(x)\\
    U_{t}(x,0) &= U_{1}(x)
\end{align*}

